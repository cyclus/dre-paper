

\begin{frontmatter}

\title{A Methodology for Determining the Dynamic Exchange of Resources in
  Nuclear Fuel Cycle Simulation}

%% Group authors per affiliation:
\author[iiasa]{Matthew J. Gidden\corref{corref}}
\cortext[corref]{Corresponding author}
\ead{gidden@iiasa.ac.at}
\address[iiasa]{International Institute for Applied Systems Analysis,
  Schlossplatz 1, A-2361 Laxenburg, Austria}
\author[uw]{Paul P. H. Wilson}
\address[uw]{University of Wisconsin - Madison, Department of Nuclear
  Engineering and Engineering Physics, Madison, WI 53706}

\begin{abstract}
Simulation of the nuclear fuel cycle can be performed using a wide range of
techniques and methodologies. Past efforts have generally been focused on
specific fuel cycles or reactor technologies. The \Cyclus fuel cycle simulator
seeks to separate the design of the simulation from the fuel cycle or
technologies of interest. In order to support this separation, a robust
supply-demand communication and solution framework is required. Accordingly
agent-based supply-chain framework, the Dynamic Resource Exchange (DRE), has
been designed implemented in \Cyclus. It supports the communication of complex
resources, namely isotopic compositions of nuclear fuel, between fuel cycle
facilities and their managers (e.g., institutions and regions). Instances of
supply and demand are defined as an optimization problem and solved for each
timestep. To display the variety of possible simulations that the DRE enables,
example scenarios are formulated and described. Important features include key
fuel-cycle facility outages, introduction of external recycled fuel sources
(similar to the current MOX Fuel Fabrication Facility in the United States), and
nontrivial interactions between fuel cycles existing in different regions.
\end{abstract}

\begin{keyword}
nuclear fuel cycle\sep  optimization\sep agent-based modeling
\end{keyword}

\end{frontmatter}

\linenumbers
