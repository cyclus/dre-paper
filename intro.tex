\section{Introduction}

The nuclear fuel cycle (NFC) is a complex, physics-dependent supply chain that
includes the full life cycle of Uranium and Thorium-based fuels, including
recycling of fuel and final internment of some subset of isotopes of transmuted
material. Uranium is mined, milled, and enriched to a variety of levels based on
the type and vintage of the reactor which is being fueled. Used fuel can then be
stored for a period of time before either being interred indefinitely or being
utilized in a advanced fuel cycle by recycling its fissile and fertile
isotopes. The ability to model such a system while maintaining physical
consistency due to transmutation and isotopic decay is a challenging simulation
problem. Through simulation, nuclear systems can be designed in order to support
decision-making processes in order to address a variety of goals, e.g., reducing
system cost, management and planning of storage facilities, studying the
dynamics governing system transitions, and estimating long-term system
sustainability.

NFC simulation is performed by a variety of actors, including governments,
universities, and international governance organizations. Accordingly, many
modeling strategies have been applied to the system. Such strategies span a wide
range of fidelity, both at the facility level and the material level. For
instance, some simulators describe reactors by fleet (or types) and solve
material balances for the entire fleet in aggregate
\cite{busquim_e_silva_system_2008,yacout_vision_2006} while others instantiate
individual (or discrete) facilities \cite{schneider_nfcsim:_2005}. Similarly,
some simulators make detailed calculations of fuel depletion due to reactor
fluence \cite{boucher_cosi:_2006} whereas others simply use pre-tabulated values
that depend (generally) on burnup values for thermal reactors and conversion
ratios for fast reactors.

% what are the primary design choices
% what have other people done

There are, broadly, three design-decision categories that are of concern to the
design of an NFC simulator. The first is facility deployment, i.e., how, why,
and when certain facilities are deployed. In the current simulation development
environment, the most common reactor deployment mechanism is allowing a user to
define an energy growth curve and, for each type of reactor in the simulation, a
percentage of that total energy demand to be met by the reactor type. It is also
common for simulators to adjust deployments based on look-ahead heuristics of
future material availability \cite{schweitzer_improved_2008,
  van_den_durpel_daness_2009}. The second decision category is the level of
fidelity with which to model the physical and chemical processes involved in the
nuclear fuel cycle. Broadly, physical fidelity includes two processes, isotopic
decay and isotopic transmutation due to residency in a reactor. To date, there
is still disagreement as to the physical fidelity required to accurately capture
all system detail \cite{guerin_impact_2009}. The third concerns the connections
between facilities and the type of material that flows along those
connections. In general, connections between facilities can either be static or
dynamic, and can either be fleet-based or facility-based. A static connection
implies that material will always flow between two types of facilities, whereas
a dynamic connection implies that a facility's input or output connection may
change. The vast majority of simulators utilize a static, fleet-based approach.

% what were goals of cyclus-style
% simulation of the NFC can be done with a variety of methods
% why decisions were made
The Cyclus NFC simulator was designed to support different levels of model
fidelity at different portions of the fuel cycle \cite{huff_cyclus_2015}. Its
design seeks to separate the concerns of the three categories described above,
supporting, for example, both fleet and individual facility models and to allow
for either exogenous or endogenous facility deployment. However, there must be a
basic infrastructure to define the method of connection and allow communication
between entities in the simulation. This infrastructure must be flexible in that
it must support each of the categories of simulation design. It must allow for
dynamic entities that enter and exit the simulation and for relationships
between those entities to change based on simulation state. Finally, it must
allow for communication of complex resource types, e.g., isotopic fuel vectors
that change with time. 

This work describes a novel approach to addressing this complicated series of
design problems. It combines methods of both discrete-event simulation and
agent-based modeling to enable communication between simulation entities with an
optimization approach to determine the constrained transfer of resources. By
Law's definition \cite{Law:1999:SMA:554952}, Cyclus is a dynamic, discrete-event
simulation that uses a fixed-increment time advance mechanism. Inspiration for
the entity communication framework was taken from the existing agent-based
supply-chain modeling literature
\cite{swaminathan_modeling_1998,julka_agent-based_2002,van_der_zee_modeling_2005,chatfield_multi-formalism_2007,holmgren_agent_2007}
which provides a natural methodological fit to the present use case. Finally, a
version of the constrained, multi-commodity transportation problem is solved.

% what did I do
% how is paper set up

The remainder of this paper is structured as follows. \secref{sec:methods}
describes in detail the communication framework, optimization problem
formulation, and possible solution techniques. \secref{meth:tariff} also
describes a new archetype in the Cyclus ecosystem that utilizes this framework
to enable entity relationships to drive material routing
decisions. \secref{sec:results} then describes a series of scenarios that
display the enhanced modeling capabilities enabled by this new simulation
framework. Finally, \secref{sec:concl} provides some concluding remarks and
observations, reflecting on potential future work and use cases.
