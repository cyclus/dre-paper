\section{Introduction}

The nuclear fuel cycle (NFC) is a complex, physics-dependent supply chain of
uranium and thorium-based fuels, including recycling of fuel and final disposal
of some subset of isotopes of transmuted material. Uranium is mined, milled, and
enriched to some level based on the type and vintage of the reactor
which is being fueled. Used fuel can then be stored for a period of time before
either being interred indefinitely or being utilized in a advanced fuel cycle by
recycling its fissile and fertile isotopes. The ability to model such a system
while maintaining physical consistency due to transmutation and isotopic decay
is a challenging simulation problem. Through simulation, nuclear systems can be
designed in order to support decision-making processes addressing a variety of
goals, e.g., reducing system cost, future planning of storage facilities,
studying the dynamics governing system transitions, and estimating long-term
system sustainability.

NFC simulation is performed by a variety of actors, including governments,
universities, and international governance organizations. Accordingly, many
modeling strategies have been applied, spanning a wide range of modeling detail
for both nuclear facilities and fuel. For instance, some simulators describe
reactors by fleet (or type) and solve material balances for the entire fleet in
aggregate \cite{busquim_e_silva_system_2008, durpel_daness_2003,
  yacout_vision_2006} while others instantiate individual (or discrete)
facilities \cite{schneider_nfcsim:_2005}. Similarly, some simulators make
detailed calculations of fuel depletion due to reactor fluence
\cite{boucher_cosi:_2006, mouginot2012class} whereas others use pre-tabulated
values that depend (generally) on burnup values for thermal reactors and
conversion ratios for fast reactors \cite{yacout_vision_2006}.

% what are the primary design choices
% what have other people done

There are, broadly, three categories of concern to the design of an NFC
simulator. The first is facility deployment, i.e., how, why, and when certain
facilities are instantiated in the simulation. The most common reactor
deployment mechanism allows a user to define an energy growth curve and, for
each type of reactor in the simulation, a percentage of that total energy demand
to be met by the reactor type. It is also common for simulators to adjust
deployments based on look-ahead heuristics of future material availability
\cite{schweitzer_improved_2008, van_den_durpel_daness_2009}. The second design
category is the fidelity with which the physical and chemical processes involved
in the nuclear fuel cycle are modeled. Broadly, physical fidelity includes two
processes, isotopic decay and isotopic transmutation due to fuel's residency in
a reactor. To date, there is still disagreement as to the physical fidelity
required to accurately capture sufficient system detail \cite{guerin_impact_2009}. The
third category concerns the communication of supply and demand between
facilities, in other words, how facilities are connected in the simulation. In
general, connections between facilities can either be static or dynamic and can
either be fleet-based or facility-based. A static connection implies that
material will always flow between two types of facilities, whereas a dynamic
connection implies that a facility's input or output connection may
change. Simulator design is dependent on the underlying modeling approach. For
example, using System Dynamics \cite{forrester1971counterintuitive} naturally
leads to a static, fleet-based approach \cite{busquim_e_silva_system_2008,
  durpel_daness_2003, yacout_vision_2006}, whereas developing a stand-alone,
discrete event or time simulation \cite{Law:1999:SMA:554952} can lead to higher
levels of modeling fidelity in areas of concern \cite{schneider_nfcsim:_2005,
  mouginot2012class, boucher_cosi:_2006}.

% what were goals of cyclus-style
% simulation of the NFC can be done with a variety of methods
% why decisions were made
\Cyclus, a NFC simulator developed by the CNERG team at the University of
Wisconsin, was designed to support different levels of model fidelity at
different portions of the fuel cycle \cite{huff_cyclus_2015}. By Law's
definition \cite{Law:1999:SMA:554952}, \Cyclus is a dynamic, discrete-event
simulation that uses a fixed-increment time advance mechanism. Its design seeks
to separate the design concerns of the three categories described above,
supporting, for example, both fleet and individual facility models and allowing
for either exogenous or endogenous facility deployment. However, a common
infrastructure defining the method of facility connection and allowing
communication between entities in the simulation is required. This
infrastructure must be flexible in order to support different approaches to each
of the categories of simulation design. To do so, it must allow for static
simulation entities (e.g., facilities) as well as dynamic entities that enter
and exit the simulation. Further, it must support the changing of relationships
between those entities based on simulation state. Finally, it must allow for
communication of complex resource types, e.g., isotopic fuel vectors that change
with time.

This work describes a novel approach to addressing this complicated series of
design problems associated with the exchange of resources in a dynamic,
physics-dependent, supply-chain simulation. It combines methods of both
discrete-event simulation and agent-based modeling with an optimization approach
to determine the constrained transfer of resources. Inspiration for the entity
communication framework was taken from the existing agent-based supply-chain
modeling literature
\cite{swaminathan_modeling_1998,julka_agent-based_2002,van_der_zee_modeling_2005,chatfield_multi-formalism_2007,holmgren_agent_2007}
which provides a natural methodological fit to the present use case. Given
time-dependent supply and demand of nuclear fuel, a version of the constrained,
multi-commodity transportation problem is solved to determine resource transfers
within a simulation time-step.

% what did I do
% how is paper set up

The remainder of this paper is structured as follows. \secref{sec:methods}
describes in detail the communication framework, optimization problem
formulation, and possible solution techniques. \secref{meth:tariff} also
describes a new archetype in the Cyclus ecosystem that utilizes this framework
to enable entity relationships to drive material routing
decisions. \secref{sec:results} then describes a series of scenarios that
display the enhanced modeling capabilities enabled by this new simulation
framework. Finally, \secref{sec:concl} provides concluding remarks and
observations, reflecting on potential future work and use cases.
