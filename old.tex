\section{\textbf{Previous Text, Use What is Needed}}

%%%%%
% Old Motivation Section
%%%%%

Fuel cycle simulation involves a number of components: modeling in-facility
processes (e.g. fuel transmutation), determining facility deployment, and
determining the flow of resources between facilities. Because Cyclus uses an
agent-based modeling approach, each component can be addressed
individually. Facility agents house in-facility process logic. Region or
Institution agents are contain the facility deployment logic. The Cyclus kernel,
informed by agents in a simulation, manages the determination of resource flow
via a multi-commodity material routing model, informed by the rich literature of
supply-chain modeling and optimization.

\subsection{The Material Routing Problem}

The multicommodity supply chain model is a well studied problem. Its goal is to
determine the flow of commodities between suppliers, consumers, and, optionally,
storage locations. A preliminary formulation has been previously explored
\cite{gidden_agent-based_2013}. When applied to the nuclear fuel cycle, this
work terms the resulting problem formulation the Material Routing Problem (MRP).

\subsubsection{Fuel Cycle Models}

Prior work in the fuel cycle simulation field has treated the problem lightly if
at all. Many simulators have simple determination logic built in to their
model. For example, system dynamics approaches generally assign each commodity a
stock and flow, and flow determination logic, if any, is translated into
cofficients of the system dynamics equations. Importantly, adding a new
commodity involves changing the underlying model, which can be a time consuming
task \cite{guerin_impact_2009}.

\subsubsection{Supply-Demand Models}

Modeling supply and demand in a supply-chain context is a field with a rich body
of literature.

Agent-based modeling approaches have also been implemented \cite{julka_agent-based_2002}.

\subsection{Fuel Cycle Concerns}

The nuclear fuel cycle has a number of concerns that must be met by a
multicommodity supply-demand formulation.

\subsubsection{Physics}

A formulation must allow for constraints based on the physics of in-facility processes. 

% eg custom constraints

\subsubsection{Fuel Forms}

A formulation must allow for the variety of forms material can take within the
nuclear fuel cycle. Specifically, any formulation must allow for quantized
material transfers, representing fuel assemblies.

% eg assemblies

\subsubsection{Institutions and Regions}

A formulation must allow for institutional and regional effects to affect the flow of resources.

% eg preferences

\subsection{Cyclus Concerns}

Cyclus must provide an abstraction that translates the supply and demand of
resource objects, e.g. \textit{Material}, from an agent context to an instance
of the MRP. 

% general API and extendable formulation


%%%%%
% Old Methods Section
%%%%%

\subsection{Modeling Supply \& Demand}

\subsubsection{Agent Communication}

An agent communication interface is used to determine supply and demand. The
communication execution is implemented in phases. First, a request for bids
(RFB) phase is executed to determine demand. Next, and response to request for
bids (RRFB) phase is executed to determine supply.

% RFB, RRFB phases

\subsubsection{Graph Representation}

Supply and demand is then translated into an \textit{ExchangeGraph}, stripping
away any derived \textit{Resource} object specialization. 

% mention separability into req/sup cases

\subsubsection{Mixed Integer-Linear Problem Representation}

In order to solve the MRP with mathematical programming solvers, a separate
translation protocol was implemented to translate instances of an
\textit{ExchangeGraph} into an instance of a mixed integer-linear (MILP) problem
using the standardized Open Solver Interface (OSI). %cite osi

\subsection{Solvers}

\subsubsection{A Greedy Heuristic}

A custom heuristic was implemented that utilizes domain-level knowledge of the
resource exchange interface in Cyclus.

\subsubsection{Linear Programming}

In the case in which partial orders are allowed (e.g., when not explicitly
modeling fuel assemblies), the MILP formulation of the MRP simplifies into a
linear program. Linear programs have very fast solution times. This work
utilizes the COIN Linear Program (CLP) solver.

\subsubsection{Mixed-Integer Programming}

% discuss heuristics, cuts, etc.
This work utilizes the COIN Branch and Cut (CBC) solver. 

\subsection{Test Cases}

Solver performance for MILPs is highly dependent on both problem structure and
coefficient values of a given instance. In the absence of a large number of
instances mined from user-run simulations, test cases must be generated. These
cases are designed to explore the behavior of the model as simulation entity
behavior becomes more precise (e.g., modeling individual assemblies) and the
number of simulation entities grows.

\subsubsection{Reactor Request Case}

A description of the RR case.

\subsubsection{Reactor Supply Case}

A description of the RS case.

\subsection{Analysis Platform}

A discussion of Cyclopts, highlighting notion of parameter space, conversion
from parameters to instances, execution, output, and analysis. 

\subsubsection{High Throughput Computing Execution}

Discuss interaction with condor, highlighting specialized nodes used for timing study.
