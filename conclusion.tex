\section{Conclusions}\label{sec:concl}

A hybrid simulation-optimization approach to the dynamic modeling of the NFC has
been presented and implemented in the \Cyclus NFC simulator. Focus has been
placed on separating the core simulator design and the associated entities
forming the simulation. Significant constraints were placed on the design of
an entity interaction mechanism. It must support arbitrary physics and chemical
constraints, as well as general supply-chain constraints, such as inventory and
processing constraints. Further, it must model the competition of resources
among entities for which demand and supply of resources may be
fungible. Finally, inter-entity interactions must be translated to the
interaction framework. The resulting interaction mechanism, termed the Dynamic
Resource Exchange (DRE), was informed chiefly from the fields of supply-chain
management, agent-based modeling, and mathematical programming.

The DRE allows agents to inform both system supply and demand of resources
through a request-bid framework. Physics fidelity is provided to agents in this
framework by utilizing fully specified \texttt{Resource} objects. For example,
nuclear fuel demand can be specified directly by an ideal isotopic vector in a
\texttt{Material} object. Once supply and demand is known, social interaction
models can be applied to affect resource flow-driving mechanisms. For example, a
tariff can be modeled by uniformly reducing preferences of transactions between
agents outside of a given \texttt{Region}. Presently, a cardinal preference
model is used as the flow-driving mechanism.

The DRE comprises three layers: a resource layer, with which agents
interact, an exchange layer, and a formulation layer. Supply, demand, and
preferences are defined in the resource layer, for a specific type of
\texttt{Resource} object. The exchange layer provides a general resource
exchange representation, irrespective of a specific object type. The
representation comprises a bipartite graph of supply and demand nodes,
supply and demand constraints, and a measure of preference for each proposed
connection between nodes. The \textit{exchange graph} generated by the DRE is
solved directly by a heuristic or translated into the NFCTP, a multicommodity
transportation problem, and solved accordingly. Resulting trades between
entities are then communicated back to the simulation.

The ability to include socioeconomic, agent-based interactions in NFC simulation
were demonstrated using the \texttt{Tariff Region}, a new entrant in the \Cyclus
ecosystem. Further, scenarios that highlight unique modeling aspects enabled by
used of the DRE were shown. The \external scenario highlighted system response
to dynamic entry and exit of an external source of recycled fuel. The \outage
case displayed how a fuel cycle system responds to unplanned individual facility
outages. Finally the \tariff scenario demonstrated how sociopolitical models can
be overlaid on top of existing fuel cycle models materially affecting scenario
results. Finally, the effects of using full optimization rather than heuristics
was investigated.

Each of the scenarios described in \secref{sec:results} intentionally highlights
one unique aspect of novel modeling capability. Critically, each new modeling
feature is modular and can be combined with any other. The core flexibility that
the DRE provides is a common communication and solution framework. Analysts
instead can focus on the specifics of their NFC core competency. A common set of
rules, specialized for each simulation entity, define its interaction mechanism,
rather than deciding \textit{a priori} exactly how entities interact. This
flexibility allows for dynamic entity exit and entry into the simulation as well
as the ability to model stochastic events.

Using a simulator that employ the DRE, e.g., \Cyclus, users may now apply
physical, economic, and social models to NFC simulation. The choice of solver
will largely depend on the fidelity of the associated models and underlying
data. The \greedy solver will always provide a feasible solution to the given
exchange instance, applying any physical, chemical, or supply-chain
constraints. Therefore, if a user has a low-fidelity economic or social model,
then the Greedy solver will likely meet the users needs. With higher-fidelity
economic and social models, obtaining a optimal solution becomes paramount. \cbc
is an available resource for solving such instances, though commercial solvers
such as CPLEX are much more computationally efficient. Solvers, in principle,
can be easily exchanged because of the use of OSI.

A novel way to model dynamic, nuclear fuel cycles has been proposed, designed,
implemented, and presented. New features include competition between suppliers
and consumers, constrained supply and consumption, and the inclusion of
extra-facility effects, such as state-level relationships. Ongoing work
continues to utilize the flexible, entity-interaction mechanism enabled by the
DRE in order to support modeling of nonproliferation applications and
medium-fidelity physics-enabled fuel cycles. The implementation of the DRE in
\Cyclus represents a significant methodological advance in NFC simulation,
supporting a variety of existing and new simulation techniques, each within a
common simulator framework.
