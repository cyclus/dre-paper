
%%%%% Packages and related options
%%
%% Note, order matters!
%%
\usepackage[acronym,toc]{glossaries}
\usepackage{graphicx}
\usepackage{booktabs} % nice rules for tables
\usepackage{microtype} % if using PDF
\usepackage{hyperref}
\usepackage{amsmath}
\usepackage{moreverb} % for verbatim snippets of code
\usepackage{fancyvrb}
\usepackage{tabularx} % for tables with line breaks
\usepackage{threeparttable} % for tables with notes
\usepackage[capitalize, noabbrev]{cleveref} % for reference multiple figures
\usepackage{calc} % allows for arithmetic on latex variables
\usepackage{float} % allows for figures to be placed explicitly
\usepackage{algorithm2e} % for algorithms
\usepackage{subfig}
\usepackage{multirow} % combining rows in tables
\usepackage{footnote}
\usepackage{titlesec} % for using \titleformat
\usepackage{slashbox} % for tables with a divided cell, see http://tex.stackexchange.com/questions/7262/diagonally-divided-table-cell
\usepackage{bashful}
\usepackage{xspace}
\usepackage{color}
\definecolor{listinggray}{gray}{0.9}
\definecolor{lbcolor}{rgb}{0.9,0.9,0.9}
\lstset{
    %backgroundcolor=\color{lbcolor},
    language={C++},
    tabsize=4,
    rulecolor=\color{black},
    upquote=true,
    aboveskip={1.5\baselineskip},
    belowskip={1.5\baselineskip},
    columns=fixed,
    extendedchars=true,
    breaklines=true,
    prebreak=\raisebox{0ex}[0ex][0ex]{\ensuremath{\hookleftarrow}},
    frame=single,
    showtabs=false,
    showspaces=false,
    showstringspaces=false,
    basicstyle=\scriptsize\ttfamily\color{black},
    keywordstyle=\color[rgb]{0,0,1.0},
    commentstyle=\color[rgb]{0.133,0.545,0.133},
    stringstyle=\color[rgb]{0.627,0.126,0.941},
    numberstyle=\color[rgb]{0,1,0},
    identifierstyle=\color{black},
    captionpos=t,
}
\lstdefinestyle{BashOutputStyle}{
  basicstyle=\small\ttfamily,
  numbers=none,
  frame=tblr,
  columns=fullflexible,
  backgroundcolor=\color{blue!10},
  linewidth=0.9\linewidth,
  xleftmargin=0.1\linewidth
}
%%%%% 

%%%%% Helpers

\newcommand{\Cycamore}{\textsc{Cycamore} }
\newcommand{\cyclus}{\textsc{Cyclus}\xspace}
\newcommand{\Cyclus}{\textsc{Cyclus} }
\newcommand{\nucl}[2]{\ensuremath{^{#1}}\mbox{#2}}
\newcommand{\code}[1]{\lstinline[basicstyle=\ttfamily\color{black}]|#1|}
\newcommand{\codeb}[1]{\texttt{#1}}
\newcommand{\units}[1] {\:\text{#1}}%

%%% detect beginning of sentences and capitalize appropriately
\sfcode`\.=1001
\sfcode`\?=1001
\sfcode`\!=1001
\sfcode`\:=1001
\newcommand\secref[1]{\ifnum\spacefactor=1001 Section \ref{#1}\else section \ref{#1}\fi}
% this seems like a limitation of sfcode (an error occurs if at the beginnig of
% a paragraph) Accordingly, this can be used manually as
% needed. http://comments.gmane.org/gmane.comp.tex.texhax/17631
\newcommand\Secref[1]{ Section \ref{#1}}
%%%
