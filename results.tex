\section{Experimentation \& Results}\label{results}

A number of computational experiments are conducted to highlight unique features
enabled by the DRE in Cyclus. Each experiment is performed by solving instances
of the DRE using both the greedy heuristic and to optimality with COINOR-CBC. A
UOX-MOX binary recycle system with all required fuel cycle facilities is taken
as the \basecase scenario in order to reduce the complexity of the fuel cycle and
highlight departures from available simulators. A simulation timeframe of 50
years is chosen with one-month timesteps (totaling 600 simulation time steps),
sufficient to display all relevant effects. The nominal parameters of all common
facilities in the simulation are shown in Appendix \ref{}. Importantly, reactors
are allowed to be fueled by either UOX or MOX, with a preference for MOX over
UOX. Spent UOX fuel is allowed to be recycled, whereas spent MOX fuel is sent
directly to a repository. In order to involve dynamicism in the simulation, the
population reactors grows linearly over time at a rate of 1 reactor every 5
years. An initial population of 20 reactors are deployed individually in each of
the first 20 timesteps of the simulation as shown in Figure
\ref{fig:deploy}. Note that deployments are staggered in the initial period in
order to avoid supply/demand clustering effect. A diagram of the full
\basecase fuel cycle is shown in Figure \ref{fig:base}.

\begin{figure}
  \begin{center}
    \includegraphics[width=\columnwidth]{rxtr_deploy.pdf}
    \caption[]{
      \label{fig:deploy}
      Reactor deployment in each simulation as a function of simulation time
      steps. Each point in the graph is a reactor being deployed in the
      simulation.}
  \end{center}
\end{figure}

\begin{figure}
  \begin{center}
    \includegraphics[width=\columnwidth]{base.pdf}
    \caption[]{
      \label{fig:base}
      Material routing between in the \basecase scenario, single-pass MOX fuel
      cycle. Possible arc flows are labeled with commodity names.}
  \end{center}
\end{figure}

Four perturbations from the \basecase scenario are used to provide examples of
modeling capability enabled through the use of the DRE. The scenarios are
summarized in Table \ref{scenarios} below and described in more detail in the
following sections.

\begin{table}[]
\centering
\caption{Short Descriptions of Scenarios Ran}
\label{scenarios}
\begin{tabularx}{\textwidth}{|p{1.5cm}|p{1.5cm}|X|X|}
\hline
\textbf{Scenario  Name} & \textbf{Scenario Handle} & \textbf{Primary Departure from Base Case}                & \textbf{Capability Highlighted}                             \\ \hline
Separations Outage      & \outage                   & Separations facility halts operation mid-simulation      & System flexibility to recycling facilities operation        \\ \hline
External MOX Supplier   & \external                 & An additional supplier of MOX enters mid-simulation      & System flexibility to entry and exit of commodity suppliers \\ \hline
Regional Tariffs        & \tariff                   & Two regions are modeled with dynamic trade relationships & Ability to model nontrivial international relationships     \\ \hline
\end{tabularx}
\end{table}

\subsection{Separations Outage: Fuel Cycles with Supply Disruption}

The DRE provides a unifying framework in which any instance of supply and demand
can be formulated and solved. This flexibility lends itself well to dynamic
simulation in which the state of actors in a simulation, by definition, can (and
will) change as the simulation progresses. In order to show case the types of
simulations that are enabled by this feature, a fuel cycle simulation is
constructed that has multiple types of reactor fuel input and a defined supply
disruption within the recycled-fuel supply chain. 

The chosen disruption is an outage of the separations facility shown in Figure
\ref{fig:base}. The outage begins at $t_i = 250$, lasts 50 time steps, and ends
at $t_f = 300$. During the outage, the remaining facilities in the supply chain
operate normally, and the flow of fuel into and out of reactors adapts according
to the state of available fresh fuel.

A comparison of the inventories of Plutonium-239 in each facility type of
interest among the \basecase and \outage scenarios is shown in Figure
\ref{fig:outage}. As can be seen in Figure \ref{fig:baseinv}, the quantity of
MOX in \reactors is under a dynamic equilibrium, oscilliating between the
maximum quanity allowable in the system and one refueling quantity less than the
maximum, based on refueling schedules. The equilibrium value increases in a
stair-step-function manner as the number of reactors increases to being able to
provide sufficient used UOX for the next marginal refueling quantity of MOX. The
quantity of MOX in \fabrication oscillates between a minimal value and a maximum
value which is sufficient for a single reactor's refueling quantity. As soon as
there is sufficient MOX fuel for another refueling \textit{and} a reactor makes
a request to be refueled, it is provided the quantity of MOX in of fresh
fuel. Finally, \separations separates the various actinides of used fuel and
passes on fissile isotopes to \fabrication, thus maintaining a small oscillating
inventory in each timestep.

\begin{figure}
  \centering
  \begin{minipage}{0.67\textwidth}
    \centering 
    \subfloat[Pu-239 inventories in the \basecase scenario.]{
      \includegraphics[height=0.3\textheight]{outage_invs_a.pdf}
      \label{fig:baseinv}}
    \vfill 
    \subfloat[Pu-239 inventories in the \outage scenario.]{
      \includegraphics[height=0.3\textheight]{outage_invs_b.pdf}
    \label{fig:outageinv}}
  \end{minipage}%
  \begin{minipage}{0.33\textwidth}
    \centering
    \subfloat[A close-up of the \outage scenario perturbation.]{
      \includegraphics[height=0.6\textheight]{outage_invs_c.pdf}
      \label{fig:outagezoom}}
  \end{minipage}%
  \caption[]{
    \label{fig:outage}
    Facility inventories of Pu-239 in \basecase and \outage scenarios.}
\end{figure}

The dynamic equilibrium behavior changes in the \outage scenario after the
initial outage time, $t_i$, as is observable in Figures \ref{fig:outageinv} and
\ref{fig:outagezoom}. Because the outage occurs in \separations, which takes
inputfrom the \reactors and provides output to \fabrication, the inventories of
both \separations and \fabrication remain constant for the duration of the
outage period. The inventory of Pu-239 in \reactors continues to oscillate
because MOX assemblies are discharged and continue to be sent to \storage,
whereas spent UOX assemblies (with significant Pu-239 inventories) are stored on
site. In the first timestep of renewed service of \separations, $t_f$, the
entirety of the pent-up store of used fuel in \reactors is send to \separations,
reducing the inventory to zero, causing the delta-function behavior in \reactor
flows seen in Figure \ref{fig:outagezoom} at $t = t_f$. \separations then
extracts all of Pu-239 in a single timestep, sending it to \fabrication and
causing the delta-function behavior in \separations flows seen in Figure
\ref{fig:outagezoom} at $t = t_f + 1$. Finally, the stock of Pu-239 in \reactors
after the outage increases due to the higher availability of MOX fuel in
\fabrication, until the dynamic equilibrium returns. The length of the full
perturbation is function of both the amount of Pu-239 required per refueling and
the number of refuelings that occurs during the outage. The more refuelings that
happen during the outage, the more excess MOX assemblies can be made, thus
continuing the dynamic equilibrium perturbation.

\subsection{External MOX Supplier: Fuel Cycles with Demand Fungibility}

The DRE allows for both positive and negative perturbations in fuel
availability. While the \outage scenario models a case where there is a
supply-chain disruption, the \external scenario models a case where
there is an injection of a preferred fuel source. An example of such a scenario
occuring in the real world includes the downblending of military-grade fuel
sources, such as the Megatons to Megawatts program and the MOX Fuel Fabrication
Facility.

In the \external scenario, a new source of MOX fuel enters halfway
through the simulation at $t = 250$, creating the fuel cycle shown in Figure
\ref{fig:military}. The total quantity of fuel the new source can provide is
limited to 10 refueling quantities (where reactors refuel one third of their
total core mass in each cycle). Preferences are assigned such that reactors
prefer MOX from its normal cycle over MOX from the new source, i.e.,
$p_{\text{MOX}} > p_{\text{MOX, new}} > p_{\text{UOX}}$. Reactors request each
of the commodities, and thus the first 10 reactors to refuel after the new
source enters the simulation when no original MOX is available will be provided
with MOX from the new facility. Reactors will continue to request fuel from the
new facility for the remainder of the simulation, but will not receive any due
to the supply constraints. This injection of a new fuel source also serves to
perturb the supply chain by delaying the amount of spent UOX available for
recycling.

\begin{figure}
  \begin{center}
    \includegraphics[width=\columnwidth]{military.pdf}
    \caption[]{
      \label{fig:military}
      Material routing between in the \external scenario fuel
      cycle. Possible arc flows are labeled with commodity names.}
  \end{center}
\end{figure}

The dynamic equilibrium of Pu-239 inventories again changes with the \external
perturbation, as shown in Figure \ref{fig:military}. A number of new features
arise, however. First, the equilibrium value during the initial transient
increases by the total quantity of refueling quantities available from the
external source of MOX (in this case 10 refueling quantities). Second, the
equilibrium value upon exiting the transient is lower than the value upon its
entrance. This is due to the fact that the amount of spent UOX in the overall
recycle system has decreased, due to the usage of external MOX, thus reducing
the availability of MOX. The system has been shocked into a new dynamic
equilibrium, with Pu-239 values slightly lower than the previous
equilibirum. This suggests that the injection of external recycled fuel can
reduce the level of which a system can sustain a recycling fuel cycle. Finally,
a small lag can be seen in the inventory of Pu-239 in \fabrication, which is due
to a loss of available spent UOX due to the increased presence of spent MOX
exiting reactors that were able to utilize external MOX. This transient is
quickly recovered from, however.

\begin{figure}
  \centering
  \begin{minipage}{0.67\textwidth}
    \centering 
    \subfloat[Pu-239 inventories in the \basecase scenario.]{
      \includegraphics[height=0.3\textheight]{military_invs_a.pdf}
      \label{fig:baseinv}}
    \vfill 
    \subfloat[Pu-239 inventories in the \external scenario.]{
      \includegraphics[height=0.3\textheight]{military_invs_b.pdf}
    \label{fig:militaryinv}}
  \end{minipage}%
  \begin{minipage}{0.33\textwidth}
    \centering
    \subfloat[A close-up of the \external scenario perturbation.]{
      \includegraphics[height=0.6\textheight]{military_invs_c.pdf}
      \label{fig:militaryzoom}}
  \end{minipage}%
  \caption[]{
    \label{fig:military}
    Facility inventories of Pu-239 in \basecase and \external scenarios.}
\end{figure}

\subsection{Regional Tariffs: Fuel Cycles with International Instruments}

One of the novel features of the DRE is the ability for different geographical
and managing entity representations to be laid over otherwise regional-agnostic
fuel cycles and affect the outcome of possible trades between those fuel
cycles. The \tariff two-region scenario showcases the ability to model
such situations.

Two regions, Region A and Region B, are modeled. Region A houses a fuel cycle
with both UOX and MOX-based fuel services, as in the \basecase scenario. The
same total number of \reactors are modeled in the scenario. Region A begins with
15 \reactors and Region B begins with 5 reactors. All reactor deployment occurs
in Region A.

In this scenario, Region A has an over-abundance of supply of fuel and can thus
provide fuel services to other regions. Region B contains a simple, once-through
fuel cycle. Although the scenario is somewhat contrived in order to highlight a
multi-commodity system under dynamic behavioral change, such fuel service
arrangements are present today in countries that provide fuel for once-through
fuel cycles \TODO{cite france/russia fuel services}. The possible flow of
commodities between fuel cycles is shown in Figure \ref{fig:region}.

\begin{figure}
  \begin{center}
    \includegraphics[width=\columnwidth]{region.pdf}
    \caption[]{
      \label{fig:region}
      A two-region set of fuel cycles separated by a dotted-red line. The upper
      region (Region A) includes a one-pass MOX fuel cycle, and the bottom
      region (Region B) includes a once-through fuel cycle connected to the
      one-pass MOX fuel cycle. Note that all spent fuel that originated in
      Region A is returned to Region A's fuel cycle.}
  \end{center}
\end{figure}

Initially, preferences are set such that fuel trade from Region A to
Region B is preferred over Region B's domestic fuel production. In other words, a
preference distribution for fuel supplied to Region B has the following
relation

\begin{equation}\label{eqn:bigdefault}
  p_{MOX, a} > p_{UOX, a} > p_{UOX, b} > 1.
\end{equation}

\noindent
This preference distribution implies that Region B's domestic fuel cycle will
never be utilized -- it will always be fueled by Region A, as long as Region A
has available capacity. 

At some time $t_0$, a time-varying tariff is applied by Region B which perturbs
preference values along arcs connecting Region A fuel suppliers with Region B
fuel consumers. Consider a tariff defined by function in Equation
\ref{eqn:tariff} with preferences adhering to the relation provided in Equation
\ref{eqn:bigp1}, which guarantees a strict preference ordering under $f(t)$.

\begin{equation}\label{eqn:tariff}
f(t)
\begin{cases}
1, & \text{if } t < t_0 \\
\frac{p_{UOX, b} - 1}{p_{UOX, a}}, & \text{if } t_0 \leq t < t_1 \\
\frac{p_{UOX, b} - 1}{p_{MOX, a}}, & \text{if } t_1 \leq t < t_2
\end{cases} 
\end{equation}

\begin{equation}\label{eqn:bigp1}
  p_{UOX, b} \left( 1 - \frac{p_{UOX, a}}{p_{MOX, a}} \right) > 1.
\end{equation}

Choosing nominal values that satisfy Equations \ref{eqn:bigdefault} and
\ref{eqn:bigp1}, e.g., $p_{MOX, a} = 9$, $p_{UOX, a} = 4$, and $p_{UOX, b} = 2$,
one arrives at actual preference values as shown in Figure \ref{fig:prefs}. In
the \tariff scenario, $t_0$ is chosen to be 150 and $t_1$ is set to
300. The DRE naturally handles the flow of commodities between Facility agents
in each Region, allowing the application of tariffs by the Region agents.

\begin{figure}
  \begin{center}
    \includegraphics[width=\columnwidth]{tariff_prefs.pdf}
    \caption[]{
      \label{fig:prefs}
      Preference values for reactors in Region B for available fuel commodities
      in Region B as a function of time.}
  \end{center}
\end{figure}

The effects of time-dependent tariff application on the simulation can be seen
in Figure \ref{fig:tariff}. The front-end of Region B's fuel cycle is not
utilized until $t > t_0$; all fuel is provided from Region A. After $t_0$, the
majority of the fuel services required by \reactors in Region B comes from
Region B's own front end. However, it is still able to utilize the MOX-based
fuel services from Region A. Finally, after the final tariff is applied at
$t_1$, Region B's \reactors stop utilizing Region A's fuel cycle entirely.    

\begin{figure}
  \begin{center}
    \includegraphics[width=\columnwidth]{tariff_b_reactor_flow.pdf}
    \caption[]{
      \label{fig:tariff}
      Cumulative flow of fuel into \reactors in Region B.}
  \end{center}
\end{figure}

\subsection{Comparisons between Scenarios}

\begin{figure}
  \centering
  \begin{minipage}{\textwidth}
    \centering
    \subfloat[A]{\includegraphics[width=0.5\textwidth]{uox_flow.pdf}}
    \subfloat[B]{\includegraphics[width=0.5\textwidth]{mox_flow.pdf}}
  \end{minipage}%
  \caption[]{
    \label{fig:prefs}
    Cumulative flow of MOX in all Scenarios.}
\end{figure}

\begin{figure}
  \centering
  \begin{minipage}{0.5\columnwidth}
    \centering
    \subfloat[A]{\includegraphics[width=\columnwidth]{pu_in_rxtrs.pdf}}
    \vfill
    \subfloat[C]{\includegraphics[width=\columnwidth]{pu_in_fabs.pdf}}
  \end{minipage}%
  \begin{minipage}{0.5\columnwidth}
    \centering
    \subfloat[B]{\includegraphics[width=\columnwidth]{pu_in_repos.pdf}}
    \vfill
    \subfloat[D]{\includegraphics[width=\columnwidth]{pu_in_fabs_zoom.pdf}}
  \end{minipage}
  \caption{Pu Inventory}\label{foo}
\end{figure}

\subsection{Solver Comparisons}

