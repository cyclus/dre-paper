\section{Motivations \& Problem Statement}

Fuel cycle simulation involves a number of components: modeling in-facility
processes (e.g. fuel transmutation), determining facility deployment, and
determining the flow of resources between facilities. Because Cyclus uses an
agent-based modeling approach, each component can be addressed
individually. Facility agents house in-facility process logic. Region or
Institution agents are contain the facility deployment logic. The Cyclus kernel,
informed by agents in a simulation, manages the determination of resource flow
via a multi-commodity material routing model, informed by the rich literature of
supply-chain modeling and optimization.

\subsection{The Material Routing Problem}

The multicommodity supply chain model is a well studied problem. Its goal is to
determine the flow of commodities between suppliers, consumers, and, optionally,
storage locations. A preliminary formulation has been previously explored
\cite{gidden_agent-based_2013}. When applied to the nuclear fuel cycle, this
work terms the resulting problem formulation the Material Routing Problem (MRP).

\subsubsection{Fuel Cycle Models}

Prior work in the fuel cycle simulation field has treated the problem lightly if
at all. Many simulators have simple determination logic built in to their
model. For example, system dynamics approaches generally assign each commodity a
stock and flow, and flow determination logic, if any, is translated into
cofficients of the system dynamics equations. Importantly, adding a new
commodity involves changing the underlying model, which can be a time consuming
task \cite{guerin_impact_2009}.

\subsubsection{Supply-Demand Models}

Modeling supply and demand in a supply-chain context is a field with a rich body
of literature.

Agent-based modeling approaches have also been implemented \cite{julka_agent-based_2002}.

\subsection{Fuel Cycle Concerns}

The nuclear fuel cycle has a number of concerns that must be met by a
multicommodity supply-demand formulation.

\subsubsection{Physics}

A formulation must allow for constraints based on the physics of in-facility processes. 

% eg custom constraints

\subsubsection{Fuel Forms}

A formulation must allow for the variety of forms material can take within the
nuclear fuel cycle. Specifically, any formulation must allow for quantized
material transfers, representing fuel assemblies.

% eg assemblies

\subsubsection{Institutions and Regions}

A formulation must allow for institutional and regional effects to affect the flow of resources.

% eg preferences

\subsection{Cyclus Concerns}

Cyclus must provide an abstraction that translates the supply and demand of
resource objects, e.g. \textit{Material}, from an agent context to an instance
of the MRP. 

% general API and extendable formulation
