%%%%%%%%%%%%%%%%%%%%%%%%%%%%%%%%%%%%%%%%%
% Plain Cover Letter
% LaTeX Template
%
% This template has been downloaded from:
% http://www.latextemplates.com
%
% Original author:
% Rensselaer Polytechnic Institute (http://www.rpi.edu/dept/arc/training/latex/resumes/)
%
%%%%%%%%%%%%%%%%%%%%%%%%%%%%%%%%%%%%%%%%%

%----------------------------------------------------------------------------------------
%       PACKAGES AND OTHER DOCUMENT CONFIGURATIONS
%----------------------------------------------------------------------------------------

\documentclass[11pt]{letter} % Default font size of the document, change to 10pt to fit more text
\usepackage{graphicx}
%\usepackage{newcent} % Default font is the New Century Schoolbook PostScript font
%\usepackage{helvet} % Uncomment this (while commenting the above line) to use the Helvetica font

% Margins
\usepackage[left=1.25in,right=1.25in,top=1.5in,bottom=1.25in]{geometry}
%\let\raggedleft\raggedright % Pushes the date (at the top) to the left, comment this line to have the date on the right

\usepackage{eso-pic,graphicx}
 \begin{document}

%----------------------------------------------------------------------------------------
%       ADDRESSEE SECTION
%----------------------------------------------------------------------------------------

\begin{letter}{Professor Yassin Hassan\\
Editor, Nuclear Engineering and Design}

%----------------------------------------------------------------------------------------
%       YOUR NAME & ADDRESS SECTION
%----------------------------------------------------------------------------------------

\address{Matthew Gidden\\
Laimgrubengasse 17/7\\
Vienna, Austria 1060}


%----------------------------------------------------------------------------------------
%       LETTER CONTENT SECTION
%----------------------------------------------------------------------------------------

\opening{Professor Hassan,}

Please find enclosed a manuscript entitled: ``A Methodology for Determining the
Dynamic Exchange of Resources in Nuclear Fuel Cycle Simulation'' which I am
submitting for exclusive consideration of publication as a full-length article
in Nuclear Engineering and Design.  This manuscript describes novel work in the
design of the fuel cycle simulator, Cyclus, highlighting a new mechanism for
entity interaction in fuel cycle simulation. Scenarios that are uniquely enabled
by the mechanism, e.g., endogenous regional relationships, are described and
analyzed.

Thank you for your consideration of this work. Please address all correspondence
concerning this manuscript to me and feel free to correspond with me by e-mail.

\closing{
Warm Regards,

Matthew Gidden, Ph.D.
}

%----------------------------------------------------------------------------------------

\end{letter}

\end{document}


